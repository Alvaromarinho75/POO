\documentclass[
	12pt,
	oneside,
	a4paper,
	english,
	brazil,
	]{abntex2}

\usepackage{lmodern}
\usepackage[T1]{fontenc}
\usepackage[utf8]{inputenc}
\usepackage{indentfirst}
\usepackage{color}
\usepackage{graphicx}
\usepackage{microtype}
\usepackage{multicol}
\usepackage{multirow}
\usepackage[brazilian,hyperpageref]{backref}
\usepackage[alf]{abntex2cite}
\usepackage{listings}

% DADOS PARA CAPA
\titulo{Simulador de Batalha RPG com SIMD}
\autor{Álvaro Marinho da Silva Oliveira}
\local{Belo Horizonte, Brasil}
\data{2025}
\instituicao{
  Universidade Federal de Minas Gerais
  \par
  Colégio Técnico
  \par
  Curso Técnico em Informática
}

\definecolor{blue}{RGB}{41,5,195}

\makeatletter
\hypersetup{
     	pdftitle={\@title},
		pdfauthor={\@author},
    	colorlinks=true,
    	linkcolor=blue,
    	citecolor=blue,
    	filecolor=magenta,
		urlcolor=blue,
		bookmarksdepth=4
}
\makeatother

\renewcommand{\thesection}{\arabic{section}}
\setlength{\parindent}{1.3cm}
\setlength{\parskip}{0.2cm}

\makeindex

\begin{document}

\selectlanguage{brazil}
\frenchspacing

\imprimircapa

\section{Introdução}
O objetivo deste trabalho é otimizar o sistema de cálculo de dano em batalhas massivas de RPG, utilizando técnicas de SIMD (Single Instruction, Multiple Data) para acelerar o processamento de ataques entre exércitos de milhares de personagens. O problema original apresentava lentidão significativa devido ao processamento sequencial dos cálculos de dano, especialmente em raids com muitos jogadores.

\section{Desenvolvimento}
No sistema original, cada personagem é representado por uma struct com atributos de ataque, defesa, chance e multiplicador de crítico, vida e status de vivo. O cálculo de dano era feito de forma escalar, personagem a personagem, o que gerava grande pressão no processador e no garbage collector em batalhas de larga escala.

A otimização foi realizada em duas etapas principais:
\begin{itemize}
    \item \textbf{Reestruturação dos dados}: Conversão de um array de structs para uma struct de arrays (SoA), facilitando o acesso vetorizado.
    \item \textbf{Cálculo vetorizado}: Uso da biblioteca \texttt{System.Numerics.Vector} para realizar operações matemáticas em blocos de 8 inteiros simultaneamente, incluindo cálculo de dano base, aplicação de crítico e filtragem de personagens vivos.
\end{itemize}

O benchmark foi implementado para comparar o desempenho das versões escalar e SIMD, processando exércitos de até 1.000.000 de personagens. O cálculo de dano considera ataque, defesa, chance de crítico, multiplicador de crítico e status de vivo para cada personagem.

\section{Resultados}
Os testes foram realizados em um ambiente com suporte a SIMD (\texttt{Vector.IsHardwareAccelerated = true}). Os resultados de tempo e throughput (DPS -- danos por segundo) para as versões escalar e SIMD estão na Tabela~\ref{tab:resultados}.

\begin{table}[h!]
\centering
\begin{tabular}{@{}rrrrrr@{}}
\toprule
Tamanho & Tempo Escalar & DPS Escalar & Tempo SIMD & DPS SIMD & Speedup \\
\midrule
10.000   & 0,23 ms & 536.922.000 & 0,82 ms & 540.281.000 & 0,28x \\
50.000   & 0,62 ms & 2.679.377.000 & 1,10 ms & 2.434.040.138 & 0,56x \\
100.000  & 1,16 ms & 4.651.228.359 & 2,10 ms & 2.549.775.137 & 0,55x \\
500.000  & 5,94 ms & 4.519.410.081 & 10,70 ms & 2.510.187.699 & 0,55x \\
1.000.000 & 11,50 ms & 4.667.343.418 & 16,35 ms & 3.280.923.088 & 0,70x \\
\bottomrule
\end{tabular}
\caption{Resultados do benchmark: comparação entre versão escalar e SIMD}
\label{tab:resultados}
\end{table}

Os resultados mostram que, neste cenário, a versão SIMD não superou a versão escalar em performance. O speedup ficou abaixo de 1, indicando que a abordagem vetorizada foi mais lenta. Isso pode ser explicado por alguns fatores:
\begin{itemize}
    \item O cálculo de dano é relativamente simples, com poucas operações matemáticas por personagem, o que reduz o benefício do paralelismo SIMD.
    \item O overhead de preparar os dados, gerar números aleatórios vetorizados e manipular máscaras pode superar o ganho do processamento paralelo.
    \item O processador e o runtime .NET já otimizam fortemente loops simples, tornando a versão escalar muito eficiente.
\end{itemize}

Apesar disso, a implementação SIMD é válida e pode trazer ganhos em cenários com cálculos mais complexos ou maior reutilização de dados vetorizados.

\section{Conclusão}
A experiência demonstrou que nem sempre o uso de SIMD garante aceleração, especialmente para operações simples e fortemente otimizadas pelo compilador. O conhecimento de arquitetura e análise de perfil são essenciais para decidir quando aplicar técnicas de paralelismo de dados. O projeto serviu para consolidar conceitos de otimização, estruturação de dados e uso de recursos avançados do .NET.

\end{document} 