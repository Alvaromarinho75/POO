\documentclass[
	12pt,				% tamanho da fonte
	oneside,			% para impressão em recto e verso. Oposto a oneside
	a4paper,			% tamanho do papel. 
	english,			% idioma adicional para hifenização
	brazil,				% o último idioma é o principal do documento
	]{abntex2}

% ---
% Pacotes fundamentais 
% ---
\usepackage{lmodern}			% Usa a fonte Latin Modern
\usepackage[T1]{fontenc}		% Selecao de codigos de fonte.
\usepackage[utf8]{inputenc}		% Codificacao do documento (conversão automática dos acentos)
\usepackage{indentfirst}		% Indenta o primeiro parágrafo de cada seção.
\usepackage{color}				% Controle das cores
\usepackage{graphicx}			% Inclusão de gráficos
\usepackage{microtype} 			% para melhorias de justificação
\usepackage{multicol}
\usepackage{multirow}
\usepackage[brazilian,hyperpageref]{backref}	 % Paginas com as citações na bibl
\usepackage[alf]{abntex2cite}	% Citações padrão ABNT
\usepackage{listings}

\lstset{language=Java,
  showspaces=false,
  showtabs=false,
  breaklines=true,
  showstringspaces=false,
  breakatwhitespace=true,
  commentstyle=\color{green},
  keywordstyle=\color{blue},
  stringstyle=\color{red},
  basicstyle=\ttfamily
}

% --- 
% CONFIGURAÇÕES DE PACOTES
% --- 

% ---
% Configurações do pacote backref
% Usado sem a opção hyperpageref de backref
\renewcommand{\backrefpagesname}{Citado na(s) página(s):~}
% Texto padrão antes do número das páginas
\renewcommand{\backref}{}
% Define os textos da citação
\renewcommand*{\backrefalt}[4]{
	\ifcase #1 %
		Nenhuma citação no texto.%
	\or
		Citado na página #2.%
	\else
		Citado #1 vezes nas páginas #2.%
	\fi}%
% ---

% ---
% Informações de dados para CAPA e FOLHA DE ROSTO
% ---
\titulo{Prática XXX\\Laboratório de YYY}
\autor{Aluno1, Aluno2, Aluno3}
\local{Belo Horizonte, Brasil}
\data{202X}
\instituicao{%
  Universidade Federal de Minas Gerais
  \par
  Colégio Técnico
  \par
  Curso Técnico em XXX}

\definecolor{blue}{RGB}{41,5,195}

\makeatletter
\hypersetup{
     	%pagebackref=true,
		pdftitle={\@title}, 
		pdfauthor={\@author},
    	pdfsubject={\imprimirpreambulo},
		colorlinks=true,       		% false: boxed links; true: colored links
    	linkcolor=blue,          	% color of internal links
    	citecolor=blue,        		% color of links to bibliography
    	filecolor=magenta,      		% color of file links
		urlcolor=blue,
		bookmarksdepth=4
}
\makeatother

\renewcommand{\thesection}{\arabic{section}}
\setlength{\parindent}{1.3cm}
\setlength{\parskip}{0.2cm} 

\makeindex


\begin{document}

\selectlanguage{brazil}
\frenchspacing 

\imprimircapa

{
\ABNTEXchapterfont

\textual

% ----------------------------------------------------------
% Introdução (exemplo de capítulo sem numeração, mas presente no Sumário)
% ----------------------------------------------------------
\section{Introdução}

O experimento realizado teve como objetivo explorar o uso de expressões regulares (REGEX) em C\#, por meio do desenvolvimento de duas aplicações: um validador de senhas e um analisador de prêmios Nobel. O validador de senhas solicita ao usuário uma senha e verifica se ela é considerada forte, de acordo com critérios pré-definidos. Já o analisador de prêmios Nobel lê um arquivo JSON contendo informações sobre os laureados e exibe os nomes dos ganhadores do prêmio de economia. O objetivo principal foi compreender a aplicação prática das expressões regulares e do processamento de arquivos JSON em C\#.

\section{Desenvolvimento}

A estratégia adotada para o validador de senhas consistiu em utilizar a classe \texttt{Regex} do .NET para verificar se a senha inserida pelo usuário atendia aos seguintes critérios: tamanho entre 7 e 16 caracteres, presença de letras maiúsculas, minúsculas, dígitos e caracteres especiais. O programa solicita repetidamente a senha até que uma senha forte seja fornecida.

No analisador de prêmios Nobel, foi utilizada a biblioteca \texttt{System.Text.Json} para deserializar o arquivo \texttt{prize.json}. O código percorre a lista de prêmios, filtra aqueles da categoria ``economics'' e exibe os nomes dos laureados. A fundamentação teórica incluiu o estudo de expressões regulares para validação de padrões e manipulação de arquivos JSON em C\#.

\begin{lstlisting}[language=C]
// Trecho do validador de senha
var regex = new Regex(@"^(?=.*[a-z])(?=.*[A-Z])(?=.*\d)(?=.*[!@#$%^&*()+=_\-{}\[\]:;""'?<>,.]).{7,16}$");
if (regex.IsMatch(password)) {
    Console.WriteLine("Sucesso! A senha é forte.");
}

// Trecho do analisador de prêmios Nobel
var root = JsonSerializer.Deserialize<RootObject>(jsonString);
var economicsLaureates = root.Prizes
    .Where(p => p.Category == "economics" && p.Laureates != null)
    .SelectMany(p => p.Laureates);
foreach (var laureate in economicsLaureates) {
    Console.WriteLine(laureate.Firstname);
}
\end{lstlisting}

\section{Resultados}

Os resultados obtidos demonstraram que o validador de senhas foi capaz de identificar corretamente senhas fortes e fracas, conforme os critérios estabelecidos. O programa forneceu feedback claro ao usuário, solicitando nova entrada quando necessário. O analisador de prêmios Nobel conseguiu ler o arquivo JSON e listar corretamente os nomes dos laureados em economia.

A análise crítica mostra que os objetivos foram atingidos: o uso de expressões regulares facilitou a validação de padrões complexos, enquanto a manipulação de JSON permitiu o processamento eficiente de dados estruturados. Não foram observados desvios significativos em relação ao esperado.

\section{Conclusão}

Os resultados alcançados estão de acordo com a teoria e os objetivos propostos. As principais dificuldades envolveram a construção da expressão regular para a senha e o mapeamento correto das classes para deserialização do JSON, ambas superadas com pesquisa e testes. O experimento proporcionou aprendizado prático sobre REGEX e manipulação de arquivos em C\#, habilidades essenciais para o desenvolvimento de aplicações robustas e seguras.

\end{document}
